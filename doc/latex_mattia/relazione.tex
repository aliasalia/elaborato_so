% !TEX encoding = UTF-8
\documentclass[a4paper, titlepage, 12pt]{article}
\usepackage[T1]{fontenc}
\usepackage[italian]{babel}
\usepackage{graphicx}
\usepackage{textcomp}
\usepackage{hyperref}
\hypersetup{
    colorlinks=true,
    linkcolor=blue,
    filecolor=magenta,      
    urlcolor=cyan,
    pdftitle={Relazione SO 2025},
}

\title{Relazione SO 2025}
\author{Alessia Gennari - VR488137, Mattia Pacchin - VR461870}
\date{08/2025}

\begin{document}
\maketitle
\tableofcontents

\section{Descrizione del progetto}

In questo progetto implementeremo le specifiche dell'opzione 2 (Pthread e FIFO su Linux / macOS).

\section{Specifiche richieste - Opzione 2}

Modello di trasferimento: il client invia al server il percorso del file e riceve l'impronta.

Per la sufficienza:
\begin{enumerate}
    \item Implementare server che riceve richieste ed invia risposte, usando FIFO
    \item Implementare client che invia richiesta e riceve risposta, usando FIFO
    \item Istanziare thread distinti per elaborare richieste multiple in modo concorrente
\end{enumerate}

Per arrivare al massimo voto:
\begin{enumerate}
    \item Schedulare le richieste pendenti in ordine di dimensione del file
    \item Introdurre un limite al numero di thread in esecuzione fissato
    \item Implementare il caching in memoria delle coppie percorso-hash già servite, così da restituire i valori computati nel caso di percorsi ripetuti
    \item Gestire richieste multiple simultanee per un dato percorso processando una sola richiesta ed attendendo il risultato nelle restanti richieste
\end{enumerate}

\section{Scelte implementative}

%\begin{figure}[!ht]
%    \includegraphics[width=\linewidth]{../img/img.png}
%    \caption{image name}
%    \label{fig:imgName}
%\end{figure}

\subsection{Librerie}
\begin{itemize}
    \item OpenSSL
\end{itemize}

\subsection{Funzionamento dettagliato}

Qua ci andrà una descrizione del funzionamento dettagliato.

\subsubsection{Questa è una sotto-sotto-sezione}

Sotto sotto.

\section{Difficoltà riscontrate}

Ancora nessuna. Speriamo continui così.

\end{document}
